% Overall, we can conclude that our results have been satisfactory. We developed three models that fit our galaxy with \(\chi _{red} ^{2}\) close to 1, being the Sérsic+exp & Sérsic (lump) our best model with \(\chi _{red} ^{2}=2.5\). Because \(\chi _{red} ^{2}\) is not 1, our model can be improved. Furthermore, if we see our residual map there are some galaxy arms that our model cannot take into account: the exponential function treats the disk as a continuum. 
% \\The primary objective of this endeavor was to learn how to perform a photometric decomposition, a goal that has been confidently and effectively realized.

% In summary, our investigation met its primary goal of unraveling the complexities of photometric decomposition, yielding encouraging results. Among the models we tested, the hybrid S\'{e}rsic plus Exponential and S\'{e}rsic (lump) configuration stood out with a reduced \( \chi_{\text{red}}^{2} \) of 2.5. While not precisely at the ideal value of 1, this measure indicates a reasonable fit. Moreover, the higher \( \chi_{\text{red}}^{2} \) can be attributed to certain structural complexities in the galaxy, such as spiral arms, not adequately handled by our mathematical framework. It's worth mentioning the customized mask, which managed to lower the \( \chi_{\text{red}}^{2} \) to 1.82, suggesting an even better fit. However, this improvement must be taken with a grain of caution: the mask was constructed in an ad-hoc manner by post-selecting the most prominent residuals, effectively giving it an advantage in minimizing \( \chi_{\text{red}}^{2} \). This 'tailoring' likely contributes to overfitting, a suspicion substantiated by the increase in AIC and BIC values. Hence, while the mask yields the best \( \chi_{\text{red}}^{2} \), its very nature tempers enthusiasm regarding its true predictive power.


In summary, our investigation met its primary goal of unraveling the complexities of photometric decomposition, yielding consistent results. Among the models we tested, the hybrid S\'{e}rsic plus Exponential and S\'{e}rsic (lump) configuration stood out with a reduced \( \chi_{\text{red}}^{2} \) of 2.5. While not precisely at the ideal value of 1, this measure indicates a reasonable fit. Moreover, the higher \( \chi_{\text{red}}^{2} \) can be attributed to certain structural complexities in the galaxy, such as spiral arms, not adequately handled by our mathematical framework. It's worth mentioning the customized mask, which managed to lower the \( \chi_{\text{red}}^{2} \) to 1.82, suggesting an even better fit. However, this improvement must be taken with caution: the mask was constructed in an ad-hoc manner by post-selecting the most prominent residuals, effectively giving it an advantage in minimizing \( \chi_{\text{red}}^{2} \). Future work could consider more intricate models specifically tailored for capturing the effects of spiral arms to improve the fit.


\section*{Data Availability Statement}
The data supporting the findings of this study, along with the code and methodologies employed, are publicly available on GitHub. For complete details on our methods and to fully reproduce our results, interested parties can refer to the following repository: \url{https://github.com/pererossello/galactic_photofit}.